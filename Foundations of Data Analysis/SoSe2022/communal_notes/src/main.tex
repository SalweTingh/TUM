\documentclass[a4paper, english, headtopline=0.08em, headsepline=0.04em, left = 1cm, right = 1cm, DIV=15]{article}



\usepackage[left=1.5cm, right=1.5cm, top=2cm, bottom=3cm]{geometry}


\usepackage[main=english]{babel}
\usepackage[T1]{fontenc} % Schrift ordentlich rendern, damit Umlaute auch Umlaute sind
\usepackage[utf8]{luainputenc} % Interpretation des Inputs in utf8 (inkl. Umlaute, Akzente, etc.)
\usepackage{lmodern} % Ordentliche skalierbare Schriftart
\usepackage{blindtext}
\usepackage{dirtytalk}
\usepackage{slashed}
\usepackage{caption} %Abb. statt Abbildung bei Grafiken sowie Tab. statt Table
\usepackage{chngcntr}
\usepackage{xcolor}

\usepackage{framed}
\usepackage{mdframed}

\usepackage{graphicx} %zum einbinden von Grafiken
\usepackage{wrapfig} %zur wrapfigure-Umgebung, damit Grafiken nicht über die ganze Seitenbreite gehen
\usepackage{listings} %to write python code
\usepackage{multicol}
\usepackage{makeidx} 
\usepackage[all]{nowidow}
\usepackage{paralist}
\usepackage{perpage}
\usepackage{tikz}
    \usetikzlibrary{mindmap,decorations.pathreplacing,arrows,calc}
\usepackage[compat=1.1.0]{tikz-feynman}
\usepackage{pgfplots}
    \pgfplotsset{compat = 1.15}
    \def\sothreecirc{\path[thin,dashed,draw=black,fill=black!20] (0,0) circle (\rad)}
    \def\dotmiddle{\fill (0,0) circle (2pt)}
    \def\rad{1.8cm}
\usepackage{todonotes}

\usepackage[separate-uncertainty = true]{siunitx}
\usepackage{nicefrac} %zum darstellen von 1/2 in "schön"
\usepackage{upgreek}
\usepackage{cancel}
\usepackage{braket}
\usepackage{tensor}
\usepackage{marvosym}
\usepackage{mathtools}
\usepackage{mathrsfs} 
\usepackage{amsfonts, amsmath, amssymb} %Mathezeugs
\usepackage{fixmath}
\usepackage{bm}
\usepackage{dsfont}
\usepackage{minted} %to highlight code syntax

\usepackage{stackengine,scalerel}
\usepackage{calc}

\usepackage{hyperref}
\makeatletter
\def\@footnotecolor{myorange}
\define@key{Hyp}{footnotecolor}{%
 \HyColor@HyperrefColor{#1}\@footnotecolor%
}
\def\@footnotemark{%
    \leavevmode
    \ifhmode\edef\@x@sf{\the\spacefactor}\nobreak\fi
    \stepcounter{Hfootnote}%
    \global\let\Hy@saved@currentHref\@currentHref
    \hyper@makecurrent{Hfootnote}%
    \global\let\Hy@footnote@currentHref\@currentHref
    \global\let\@currentHref\Hy@saved@currentHref
    \hyper@linkstart{footnote}{\Hy@footnote@currentHref}%
    \@makefnmark
    \hyper@linkend
    \ifhmode\spacefactor\@x@sf\fi
    \relax
  }%
\makeatother
\hypersetup{
  linkcolor = blue,
  citecolor  = myorange,
  urlcolor   = myblue,
  colorlinks = true,
}
%
\definecolor{myred}  {HTML}{A3061E}
\definecolor{myblue} {RGB} {0,63,119}
\definecolor{myyellow} {cmy} {0,0.263,0.741}
\definecolor{mygreen} {HTML}{0B6E4F}
%
\colorlet{myorange} {myyellow!60!myred}
\colorlet{myviolett} {myred!50!myblue!80}
\renewcommand{\thefootnote}{\roman{footnote}}
\MakePerPage{footnote} %Fußnoten werden am Ende jeder Seite statt am Ende des Dokuments angezeigt


%\setcounter{tocdepth}{1} %Im Inhaltsverzeichnis werden Chapter und Sections angezeigt.

\newtheorem{theorem}{Theorem}[section]
\newtheorem{corollary}{Corollary}[theorem]
\newtheorem{lemma}[theorem]{Lemma}
\newtheorem{remark}{Remark}[section]

\newlength\shlength

\setlength{\parindent}{0pt}
\setlength{\parskip}{0.5\baselineskip}



\definecolor{shadecolor}{RGB}{192,192,192}

\DeclareMathOperator*{\argmax}{arg\,max}
\DeclareMathOperator*{\argmin}{arg\,min}
\newcommand{\ti}{\text{i}}
\newcommand{\td}{\text{d}}
\newcommand{\te}{\text{e}}
\newcommand{\prt}{\partial}
\newcommand{\hilb}{\mathcal{H}}
\newcommand{\OO}{\mathcal{O}}
\newcommand{\CC}{\mathds{C}}
\newcommand{\RR}{\mathds{R}}
\newcommand{\PP}{\mathds{P}}
\newcommand{\KK}{\mathds{K}}
\newcommand{\NN}{\mathds{N}}
\newcommand{\epsi}{\varepsilon}
\newcommand{\epso}{\varepsilon_0}
\newcommand{\vphi}{\varphi}
\newcommand{\inte}{\int\limits}
\renewcommand{\(}{\left(}
\renewcommand{\)}{\right)}
\renewcommand{\[}{\left[}
\renewcommand{\]}{\right]}
\newcommand{\lb}{\left(}
\newcommand{\rb}{\right)}
\newcommand{\lsb}{\left[}
\newcommand{\rsb}{\right]}
\newcommand{\llb}{\left\lbrace}
\newcommand{\rrb}{\right\rbrace}
\newcommand{\llangle}{\left<}
\newcommand{\rrangle}{\right>}
\newcommand{\rra}{\rrangle}
\newcommand{\lla}{\llangle}
\newcommand{\labs}{\left|}
\newcommand{\rabs}{\right|}
\newcommand{\lno}{\left.}
\newcommand{\rno}{\right.}
\newcommand{\ub}[2]{\,\smash{\underbrace{#1}_{#2}}\vphantom{#1}\,}
\newcommand{\lt}{\leadsto}
\newcommand{\ra}{\rightarrow}
\newcommand{\Ra}{\Rightarrow}
\newcommand{\ua}{\uparrow}
\newcommand{\Ua}{\Uparrow}
\newcommand{\lra}{\leftrightarrow}
\newcommand{\Lra}{\Leftrightarrow}
\newcommand{\da}{\downarrow}
\newcommand{\Da}{\Downarrow}
\newcommand{\cra}{\curvearrowright}
\renewcommand{\braket}{\Braket}
\renewcommand{\bra}{\Bra}
\renewcommand{\ket}{\Ket}
\newcommand{\mel}{\matrixelement}
\renewcommand{\set}{\Set}
\renewcommand{\bar}{\overline}
\newcommand{\Pa}{\mathrm{d}}
\renewcommand{\mathbb}[1]{\mathds{#1}}
\newcommand{\pa}{\partial}
\newcommand{\defi}{\coloneqq} % :=
\newcommand{\Chi}{\mathcal{X}}
\newcommand{\dom}{\mathrm{dom}}
\newcommand{\mr}[1]{\mathrm{#1}}
\newcommand{\mf}[1]{\mathfrak{#1}}
\newcommand{\mc}[1]{\mathcal{#1}}
\newcommand{\mb}[1]{\mathbb{#1}}
\newcommand{\ms}[1]{\mathscr{#1}}
\newcommand{\tb}[1]{\textbf{#1}}
\newcommand{\tk}[1]{\textit{#1}}
\newcommand{\ul}[1]{\underline{#1}}
\renewcommand{\vec}[1]{\mathbold{#1}}
\newcommand{\define}{\coloneqq} % :=
\newcommand{\thor}{\Lightning{}}
\newcommand{\Hil}{\mathscr{H}}
\newcommand{\1}{\mathbbm{1}}
\newcommand{\Lag}{\mc{L}}
\newcommand{\E}{\mc{E}}
\newcommand{\Ham}{\mc{H}}
\newcommand{\Lap}{\pmb{\Delta}}
\newcommand{\tr}{\textbf{tr}}
\newcommand{\Var}{\textbf{Var}}
\newcommand{\sgn}{\textbf{sgn}}
\newcommand{\grad}{\textbf{grad}}
\newcommand{\rot}{\textbf{rot}}
\newcommand{\quabla}{\pmb{\square}}
\newcommand{\nab}{\pmb{\nabla}}
\renewcommand{\dfrac}[2]{\frac{\Pa #1}{\Pa #2}}
\newcommand{\delfrac}[2]{\frac{\pa #1}{\pa #2}}
\newcommand{\imp}[1]{\textbf{\color{blue!80!black}#1\color{black}}\\}
\newcommand{\mar}[1]{\textbf{#1}\\}
\newcommand{\J}{\mc{J}}
\newcommand{\RM}[1]{\MakeUppercase{\romannumeral #1{.}}}
\newcommand{\nbox}[1]{\fbox{\parbox{.5\textwidth}{#1}}}

\newcounter{lecturesc}

\newcommand{\lecture}[1]{{\refstepcounter{lecturesc} \par% Doppelte Klammern wichtig, damit graue Farbe eingeschränkt wird
  \color{gray}
  \bigskip%
  \hrule height 0.5pt%
  \kern 5pt%
  \hbox to \textwidth{\hfil\small\smash{End of lecture \thelecturesc\ on \textbf{#1}}\vphantom{M}\hfil}%
  \kern 5pt%
  \hrule height 0.5pt%
  \kern\medskipamount%
}} % Doppelte Klammern wichtig, damit graue Farbe eingeschränkt wird
%\renewcommand{\vorlesung}[1]{} %Einkommentieren, um Vorlesungsdaten auszublenden

\newenvironment{lastlecture}[1]{\begin{shaded*}
\begin{flushleft}
\textbf{Last Lecture: #1 }
\end{flushleft}
\begin{flushleft}}
{\end{flushleft}
\end{shaded*}}

\newmdenv[
  topline=false,
  bottomline=false,
  rightline = false,
  linewidth = 3pt,
  linecolor = gray,
  skipabove=\topsep,
  skipbelow=\topsep
]{siderules}

\newcommand{\arem}[1]{\begin{leftbar} #1 \end{leftbar}}
% template from https://latex.tum.de/project/618c4a584a8e454c5f4cfa1a
\begin{document}
{
    \begin{titlepage}
    	\centering
    	\vfill
    	{\scshape\LARGE Technische Universität München \par}
    	\vfill
    	{\scshape\Large Summary of the lecture MA4800\\   \par}
    	{\huge\bfseries Foundations in Data Analysis \par}
    	\vfill
    	{\Large\itshape Instructors: Prof. Felix Krahmer and Dr. Anna Veselovska \par}
    	\vfill
    \end{titlepage}
}

\tableofcontents
\clearpage 

% section Linear Algebra Review
%%%%%%%%%%%%%%%%%%%%%%%%%%%%%%%%%%%%%%%%%
%%%%%%%%%%%%% SECTION START %%%%%%%%%%%%%
%%%%%%%%%%%%%%%%%%%%%%%%%%%%%%%%%%%%%%%%%
\section{Linear Algebra Review}
\subsection{The setup}
\begin{itemize}
	\item We work on $\mathbb{K} \in\{\mathbb{R}, \mathbb{C}\}$.
	\item $A^H = \overline{(A^T)}$.
	\item A Hermitian matrix $A$ satisfies $A=A^H$.
	\item $A^{(i)}$ are rows and $A_{(j)}$ are the columns.
	\item $A^{(i)}=\left(a_{i j}\right)_{j \in J}$ 
	and $A_{(j)}=\left(a_{i j}\right)_{i \in I}=\left(A^{T}\right)^{(j)}$
	\item The matrix-vector product between 
	$A \in \mathbb{K}^{I \times J}$ and $x \in \mathbb{K}^{I}$ results in the vector in $A x \in K^{\PP ime}$ with entries
\end{itemize}
\subsection{Matrices}
\begin{align*}
	(A x)_{i}=\sum_{j \in J} a_{i j} x_{j} .
\end{align*}
\subsection{Matrix multiplication}
The matrix-matrix product between $A \in \mathbb{K}^{I\times J}$ and $B \in \mathbb{K}^{J \times L}$ yields the matrix in $\mathbb{K}^{I \times L}$ with entries
\begin{align*}
	(A B)_{i \ell}=\sum_{j \in J} A_{i j} B_{j \ell} .
\end{align*}
% section Linear Algebra Review
%%%%%%%%%%%%%%%%%%%%%%%%%%%%%%%%%%%%%%%%%
%%%%%%%%%%%%% SECTION END %%%%%%%%%%%%%
%%%%%%%%%%%%%%%%%%%%%%%%%%%%%%%%%%%%%%%%%

% section The Singular Value Decomposition
%%%%%%%%%%%%%%%%%%%%%%%%%%%%%%%%%%%%%%%%%
%%%%%%%%%%%%% SECTION START %%%%%%%%%%%%%
%%%%%%%%%%%%%%%%%%%%%%%%%%%%%%%%%%%%%%%%%
\section{The Singular Value Decomposition}
\subsection{The leading singular vector}
\subsection{Principal components}
\subsection{Further singular vectors}
\subsection{Best k-rank approximation}

\subsection{The power method}
\begin{lemma} \label{lemma6.1} %named same with the numbers in the slides
Let $x \in \mathbb{R}^{d}$ be a unit $d$-dimensional vector of components $x=$ $\left(x_{1}, \ldots, x_{d}\right)$ with respect to the canonical basis and picked uniformly at random from the sphere $\left\{x:\|x\|_{2}=1\right\}$. The probability that $\left|x_{1}\right| \geq \alpha>0$ is at least $1-C \alpha \sqrt{d}$ for some absolute constant.
\end{lemma}
\subsubsection*{Proof}
We want the probability of y picked uniformly at random from
\begin{align*}
	B^d(1)=\left\{y\in\RR^d,||y||_2 \leq 1\right\}
\end{align*}	
satisfies $|y_1|>\alpha$. In other words, we are looking for the fraction of $B^d(1)$ that satisfies $|y_1|>\alpha$.
This corresponds to
\begin{align*}
	V_{\alpha}:=\text{Vol}(B^d(1) \cap \{y : |y_1| \leq \alpha\})
\end{align*}
\begin{align*}
	=\int_{y\in B^d(1) \cap \{y : |y_1| \leq \alpha\}}1dy
\end{align*}
\begin{align*}
	=\int_{-\alpha}^{\alpha} \left(\int_{\RR^{d-1}} 1_{y_2^2+...+y_d^2\leq 1-y_1^2}\,dy_2...dy_d\right)dy_1
\end{align*}
\begin{align*}
=\int_{-\alpha}^{\alpha} \text{Vol}\left(B^{d-1}\left(\sqrt{1-y_1^2}\right)\right)dy_1
\end{align*}
Replacing $\text{Vol}\left(B^{d-1}\left(\sqrt{1-y_1^2}\right)\right)$ with $(\sqrt{1-y_1^2})^{d-1}\text{Vol}\left(B^{d-1}(1)\right)$ since the volume the unit ball with a factor proportional to radius in the power of $d-1$.
\begin{align*}
	=\int_{-\alpha}^{\alpha} (\sqrt{1-y_1^2})^{d-1}\text{Vol}\left(B^{d-1}(1)\right)dy_1
\end{align*}
\begin{align*}
	=\text{Vol}\left(B^{d-1}(1)\right)\int_{-\alpha}^{\alpha} (1-y_1^2)^{(d-1)/2}dy_1
\end{align*}
In the integral part, $\int_{-\alpha}^{\alpha} (1-y_1^2)^{(d-1)/2}dy_1$, notice that $(1-y_1^2)^{(d-1)/2}<1$ in the whole integration domain.
Thus we can write
\begin{align*}
	=\text{Vol}\left(B^{d-1}(1)\right)\int_{-\alpha}^{\alpha} (1-y_1^2)^{(d-1)/2}dy_1
\end{align*}
\begin{align*}
	\leq \text{Vol}\left(B^{d-1}(1)\right)\int_{-\alpha}^{\alpha} 1dy_1
\end{align*}
\begin{align*}
	=2\alpha \text{Vol}\left(B^{d-1}(1)\right)
\end{align*}
Recall that volume of unit ball in d dimensions is asymptotically
\begin{align*}
	V_1 = \frac{1}{\sqrt{d\pi}}\left(\frac{2 \pi e}{d}\right)^{d/2}
\end{align*}
Hence the probability $p=\PP (\alpha \leq|y_1|)$ we are interested in satisfies asymptotically
\begin{align*}
	p=\frac{V_\alpha}{V_1} \PP opto
\frac{2\alpha\frac{1}{\sqrt{(d-1)\pi}}\left(\frac{2 \pi e}{d-1}\right)^{(d-1)/2}}
 {\frac{1}{\sqrt{d\pi}}\left(\frac{2 \pi e}{d}\right)^{d/2}}
=
\frac{2\alpha\frac{1}{\sqrt{(d-1)\pi}}\left(\frac{2 \pi e}{d-1}\right)^{(d-1)/2}}
{\frac{1}{\sqrt{d\pi}}\left(\frac{2 \pi e}{d}\right)^{(d-1)/2}\left(\frac{2 \pi e}{d}\right)^{1/2}}
\end{align*}
We simplify the last term
\begin{align*}
	=2\alpha* \left(\frac{d}{d-1}\right)^{1/2} *   \left(\frac{d}{d-1}\right)^{(d-1)/2} * \left(\frac{d}{2\pi e}\right)^{1/2}
\end{align*}
\begin{align*}
	=2\alpha*  \left(\frac{d}{\sqrt{2\pi e (d-1)}}\right) *   \left(\frac{d}{d-1}\right)^{(d-1)/2}
\end{align*}
 Since $\frac{d}{d-1} = 1+\frac{1}{d-1}$
\begin{align*}
	=2\alpha*  \left(\frac{d}{\sqrt{2\pi e (d-1)}}\right) *   \left(1+\frac{1}{d-1}\right)^{(d-1)/2}
\end{align*}
We modify the power of the same term, to show it as
\begin{align*}
	=2\alpha*  \left(\frac{d}{\sqrt{2\pi e (d-1)}}\right) *   \left(\left(1+\frac{1}{d-1}\right)^{(d-1)}\right)^{1/2}
\end{align*}
Recall that 
\begin{align*}
	e = \lim_{n \rightarrow \infty} \left(1+1/n\right)^n
\end{align*}
Thus this term is bounded with $\sqrt{e}$
\begin{align*}
	\leq 2\alpha*  \left(\frac{d}{\sqrt{2\pi e (d-1)}}\right) *   \sqrt{e}
\end{align*}
We reformulate as
\begin{align*}
	= \alpha \sqrt{d}\sqrt{\frac{2d}{\pi(d-1)}}
\end{align*}
Since $\sqrt{\frac{d}{d-1}}\leq 2$ for $d\geq 2$
\begin{align*}
	\leq \frac{2\sqrt{2}}{\pi}\alpha \sqrt{d} 
\end{align*}
Given that all of this only holds asymptotically; we might need another multiplicative constant to make it hold in general. Hence the constant $C$ in the theorem.
\begin{align*}
	p \leq  C\alpha \sqrt{d}
\end{align*}
This bounds the probability $p = \PP (\alpha \leq|y_1|) \leq  C\alpha \sqrt{d}$. Considering the probability of the complement event
the bounds $1-\PP (\alpha > |y_1|) \leq  C\alpha \sqrt{d}$ can be stated as 
\begin{align*}
	1-C\alpha \sqrt{d} \leq \PP (\alpha > |y_1|).
\end{align*}
\begin{remark} \label{remark6.2}
Notice that in the previous result essentially shows also that, independently of the dimension d, the $x_1 = \langle x,u_1\rangle$  component
of a random unit vector x with respect to any orthonormal basis $\{u_1 , . . . , u_d \}^1$ is bounded away from zero with overwhelming
probability.
\end{remark}
\begin{remark} \label{remark6.3}
Consider the isometric mapping $(a,b) \rightarrow a+bi$ from $\RR^2$ to $\CC$. The previous
result extends to random unit vectors in $\CC^d$ simplify by modifying the statement as follows:
The probability that, for a randomly chosen unit vector $z\in\CC^d$, $|z_1|\geq\alpha >0$ holds is
at least $1-C\alpha\sqrt{2d}=1-C'\alpha\sqrt{d}$.
\end{remark}
It is important to note that \cref{remark6.2} and \cref{remark6.3} holds with any orthonormal basis by rotating it to coincide


with the canonical basis.
\begin{theorem}
Let $A\in \KK^{I \times  J}$ and $x\in \KK^I$. Let $V$ be the space spanned by
the left singular vectors of $A$ corresponding to singular values greater than $(1-\epsilon)\sigma_1$.
Let $m\in \Omega\left(\frac{\ln(d/\epsilon)}{\epsilon}\right)$. Let $w^*$ be the unit vector after m iterations
of the power method, namely,
\begin{align} \label{t6.4_theorem}
	w^* = \frac{\left(AA^H\right)^mx}{||\left(AA^H\right)^mx||_2}
\end{align}
The probability that $w^*$ has a component of at most $l$, where  $l \in O\left(\frac{\epsilon}{\alpha d}\right)$,
orthogonal to $V$ is at least $1-C\alpha\sqrt{d}$ i.e. $1-C\alpha\sqrt{d} < \PP \left(\|\text{Proj}_{V^{\bot }}(w^*)\|_2<l\right)$.
\end{theorem}
% todo: what does this mean give a more meaning full explanation and an example
\subsubsection*{Proof}
% todo refer to SVD
Let the SVD of $A$ be given by
\begin{align*}
	A = \sum_{k=1}^r \sigma_k u_k v_k^H
\end{align*}
If the rank of A is less than $n=|I|$ we complete the orthonormal set of vectors $\left\{u_1,...,u_r\right\}$
into a full orthogonal basis $\left\{u_1,...,u_n\right\}$ of the $n$-dimensional space.
% todo: maybe orthonormal basis?
We can expand $x$ in the terms of this basis as
\begin{align*}
	x = \sum_{k=1}^n \langle x,u_k\rangle u_k
\end{align*}
% todo: where is this expansion mentioned before
We set $\sigma_k=0$ for $k>r$ so that we can write $A$ as
\begin{align*}
	A = \sum_{k=1}^n \sigma_k u_k v_k^H
\end{align*}
It follows that 
% todo: refer to this equation
\begin{align*}
	(AA^H)^m x = \sum_{k=1}^n \sigma_k^{2m}u_ku_k^Hx = \sum_{k=1}^n \sigma_k^{2m}u_k\langle x,u_k\rangle
\end{align*}
By \cref{lemma6.1}, \cref{remark6.2} and \cref{remark6.3} one 



has $|\langle x_1,u_1 \rangle| \geq  \alpha > 0$ with probability 
at least $1-C\alpha \sqrt{d}$.
We choose $r_{\epsilon}$ such that $\sigma_1,...,\sigma_{r_\epsilon}$ are the singular values of
$A$ that are greater or equal to $(1-\epsilon)\sigma_1$ and $\sigma_{r_{\epsilon}+1},...,\sigma_n$ are those 
that are less than $(1-\epsilon)\sigma_1$. Notice that $V = \text{span}\left\{\sigma_1,...,\sigma_{r_\epsilon}\right\}$ and
$V^\bot = \text{span}\left\{\sigma_{r_{\epsilon+1}},...,\sigma_{n}\right\}$.
The component of $w^*$ orthogonal to $V^\bot$ is $\text{Proj}_{V^{\bot }}(w^*)$ which can be written as
\begin{align}\label{t6.4_proj}
	\text{Proj}_{V^{\bot }}(w^*) = \frac{\text{Proj}_{V^{\bot}}\left( \left( AA^H\right)^m x \right)}{\|\left(AA^H\right)^m x\|_2}
\end{align}
We find  denominator of \cref{t6.4_proj} by Pythagoras-Fourier theorem

\begin{align} \label{t6.4_denom}
	||(AA^H)^m x||_2^2 = \sum_{k=1}^{n}\sigma_k^{4m} |\langle x,u_k\rangle|^2
\end{align}
\begin{align} \label{t6.4_denom_bound}
	\sum_{k=1}^{n}\sigma_k^{4m} |\langle x,u_k\rangle|^2 \geq
	\sigma_1^{4m}|\langle x,u_1\rangle|^2 \geq
	\sigma_1^{4m}\alpha^2 
\end{align}
with probability at least $1-C\alpha\sqrt{d}$. To find the nominator of \cref{t6.4_proj}, 

we check component of $(AA^H)^m x$ that is orthogonal to $V = span\{u_1,\dots,u_{r_\epsilon}\}$, namely,
\begin{align} \label{t6.4_nom}
	\text{Proj}_{V^{\bot}}\left( \left( AA^H\right)^m x \right) = \sum_{k=1}^n \sigma_k^{2m} |\langle x, u_k \rangle|_2=\sum_{k=r_\epsilon + 1}^n \sigma_k^{2m} |\langle x, u_k \rangle|_2 
\end{align} 
\begin{align} \label{t6.4_nom_bound}
	\text{Proj}_{V^{\bot}}\left( \left( AA^H\right)^m x \right) 
	\leq (1-\epsilon)^{2m} \sigma_1^{2m} \sum_{k=r_\epsilon + 1}^n |\langle x, u_k\rangle|_2 \leq (1-\epsilon)^{2m} \sigma_1^{2m}	
\end{align}
since $\sum_{k=r_\epsilon + 1}^n \leq ||x||_2^2 = 1$ and $(1-\epsilon)\sigma_1>\sigma_k$ for $r_{\epsilon}<k$.

By using \cref{t6.4_denom} and \cref{t6.4_nom} we find squared norm of the component of $w^*$ orthogonal to $V$, that is $\|\text{Proj}_{V^{\bot }}(w^*)\|_2^2$, as


$$
\|\text{Proj}_{V^{\bot }}(w^*)\|_2^2
= \frac{\sum_{k=r_\epsilon + 1}^n \sigma_k^{4m} |\langle x, u_k \rangle|^2}{\sum_{k=1}^n \sigma_k^{4m} |\langle x, u_k \rangle|^2}
$$
We bound this term by using the relations \cref{t6.4_denom_bound} and \cref{t6.4_nom_bound}


$$
\|\text{Proj}_{V^{\bot }}(w^*)\|_2^2=
\frac{\sum_{k=r_\epsilon + 1}^n \sigma_k^{4m} |\langle x, u_k \rangle|^2}{\sum_{k=1}^n \sigma_k^{4m} |\langle x, u_k \rangle|^2}
\leq \frac{(1-\epsilon)^{4m} \sigma_1^{4m}}{\alpha^2 \sigma_1^{4m}}
= \frac{(1-\epsilon)^{4m}}{\alpha^2}
$$
Thus, by taking the square root we have
\begin{align*}
	\|\text{Proj}_{V^{\bot }}(w^*)\|_2 \leq \frac{(1-\epsilon)^{2m}}{\alpha}
\end{align*}
In terms of \textit{Big O} notation we have
\begin{align*}
	\|\text{Proj}_{V^{\bot }}(w^*)\|_2 \in \OO\left(\frac{(1-\epsilon)^{2m}}{\alpha}\right) 
\end{align*}
Notice that $1-\epsilon$ is a linear approximation of $e^{-\epsilon}$. Similarly, $(1-\epsilon)^{2m}$ approximates
$e^{-2m}$ for small $\epsilon$. Using this approximation,
\begin{align*}
	\|\text{Proj}_{V^{\bot }}(w^*)\|_2 \in \OO\left(\frac{e^{-2\epsilon m}}{\alpha}\right) 
\end{align*}
Recall that $m\in \Omega\left(\frac{\ln(d/\epsilon)}{\epsilon}\right)$. This is another way
of saying there exists $m_0$ and some constant $c>0$ such that $m \geq c\frac{\ln(d/\epsilon)}{d}$ for all $m>m_0$.
Similarly, this also means there exists $m_0$ and some 
constant $c>0$ such that $-\frac{m}{c} \leq -\frac{\ln(d/\epsilon)}{d}$ for all $m>m_0$.
Since exponentiation is a non-decreasing function $e^{-\frac{m}{c}} \leq e^{-\frac{\ln(d/\epsilon)}{\epsilon}} = e^{\frac{\ln(\epsilon/d)}{\epsilon}} = (\epsilon/d)^{1/\epsilon}$. We have
\begin{align*}
	e^{-\frac{m}{c}} \leq (\epsilon/d)^{1/\epsilon}
\end{align*}
for some constant $c>0$ and $m>m_0$. 
We take the power of $\epsilon$ of both sides
\begin{align*}
	e^{-\frac{m\epsilon}{c}} \leq \frac{\epsilon}{d}
\end{align*}
Let $c_1 = c/2$
\begin{align*}
	e^{-\frac{2m\epsilon}{c_1}} \leq \frac{\epsilon}{d}
\end{align*}
For some constant $e^{-1/c_1}>0$ and all $m > m_0$. We divide both sides with $\alpha$
\begin{align*}
	\frac{e^{-\frac{2m\epsilon}{c_1}}}{\alpha} \leq \frac{\epsilon}{\alpha d}
\end{align*}
\begin{align*}
	{\alpha}^{-1}{e^{-\frac{2m\epsilon}{c_1}}} \leq \frac{\epsilon}{\alpha d}
\end{align*}
Which means
\begin{align*}
	\frac{e^{-{2m\epsilon}}}{\alpha} \in \OO\left(\frac{\epsilon}{\alpha d}\right) 
\end{align*}
Consequently
\begin{align*}
	\|\text{Proj}_{V^{\bot }}(w^*)\|_2 \in \OO\left(\frac{\epsilon}{\alpha d}\right) 
\end{align*}
%todo ref pythagoras fourier theorem
\subsection{Stability of the Singular Value
Decomposition}
Common scenario: data matrix of interest $A$, but one has
only a perturbed version of it $\tilde{A} = A + E$ where $E$ is going
to be a relatively small quantifiable perturbation.

We start by considering Hermitian matrices $A\in \KK^{I \times I}$, i.e.
$A = A^H$.
Recall that a nonzero vector $v\in K^{I}$ is an eigenvector of $A$
if $Av = \lambda v$ for some scalar $\lambda\in \KK$ called the 
corresponding eigenvalue.  The following theorem establishes that Hermitian
matrices have real eigenvalues and orthogonal
eigenvectors.
\begin{theorem}[Spectral theorem for Hermitian matrices]
If $A \in K^{I \times I}$ and $A = A^H$ , then there exists an orthonormal basis
$\{v_1, ..., v_n\}$ consisting of eigenvectors of A with real corresponding 
eigenvalues $\{\lambda_1, ..., \lambda_n\}$ such that
	\begin{align*}
		A = \sum_{k=1} \lambda_k v_k {v_k}^H
	\end{align*}
	This representation is called the spectral decomposition of $A$.
\end{theorem}
As a consequence of the spectral theorem, for Hermitian
matrices, singular value decomposition and eigenvalue
decomposition are closely related.
Indeed, by denoting $u_{k}=v_{k}\operatorname{sign}\lambda_{k}$ and $\sigma_{k}=|\lambda_{k}|$, then
$$A=\sum_{k=1}^{n}\lambda_{k}v_{k}v_{k}^{H}=\sum_{k=1}^{n}\mathrm{sign}\,\lambda_{k}|\lambda_{k}|v_{k}v_{k}^{H}=\sum_{k=1}^{n}\sigma_{k}u_{k}v_{k}^{H},$$
which is actually the SVD of $A$. Thus the SVD agrees with the spectral decomposition up
to signs. A first step towards analyzing stability is hence to study
stability of the spectral decomposition.
\subsubsection*{Weyl's Bounds}
Assume $\tilde A = {\tilde A}^H$ is Hermitian and hence also $E$.
As the eigenvalues of $A$ are real, we can order in a
non-increasing order $\lambda_{1}\geq\,\lambda_{2}\geq\,\cdot\,\cdot\,\geq\,\lambda_{n}$.
Recall that we cannot assume that the eigenvalues are positive.
Let's show
\begin{align*}
	\lambda_1(\tilde A) = \max_{|v|^2_{2}=1}v^{H}(\tilde A)v 
\end{align*}
Let $v=\tilde v_1$ from the spectral decomposition of $\tilde A$.
\begin{align*}
	 \tilde {v_1}^H \tilde A \tilde {v_1}  \leq \max_{|v|^2_{2}=1}v^{H}(\tilde A)v \\
	 \tilde \lambda_1  \leq \max_{|v|^2_{2}=1}v^{H}(\tilde A)v 
\end{align*}
Now consider the maximizer $v^*$ decomposed into orthonormal basis vectors i.e. $v^* = \sum_j\alpha_j\tilde v_j$ where $\sum_j \alpha_j^2 = 1$.
Let's plug it in
\begin{align*}
	\max_{|v|^2_{2}=1} \left(\sum_j{\bar\alpha_j}\tilde v_j^H\right)\tilde A\left(\sum_j\alpha_j\tilde v_j\right)
\end{align*}
Let $\tilde A = \sum_j \tilde \lambda_j \tilde v_j \tilde v_j^H$ be the spectral decomposition. Let's plug it in the equation.
\begin{align*}
	\max_{|v|^2_{2}=1} \left(\sum_j{\bar\alpha_j}\tilde v_j^H\right) \sum_j \tilde \lambda_j \tilde v_j \tilde v_j^H\left(\sum_j\alpha_j\tilde v_j\right) \\
\end{align*}
When the orthogonal vectors cancel out we will have the following
\begin{align*}
	\max_{|v|^2_{2}=1} \sum_j \alpha_j^2\tilde\lambda_j
\end{align*}
Notice that 
\begin{align*}
\sum_j \alpha_j^2 \tilde \lambda_j \leq \sum_j  \alpha_j^2 \tilde \lambda_1 = \tilde \lambda_1
\end{align*}
since $\sum_j \alpha_j^2 = 1$. Hence we have
\begin{align*}
	\lambda_1(\tilde A) = \max_{|v|^2_{2}=1}v^{H}(\tilde A)v 
\end{align*}
\begin{align*}
	\lambda_1(\tilde A) = \max_{|v|^2_{2}=1}v^{H}(A+E)v \\
\end{align*}
\begin{align*}
	\max_{|v|^2_{2}=1}v^{H}(A+E)v \leq \max_{|v|^2_{2}=1}v^{H}(A)v + \max_{|v|^2_{2}=1}v^{H}(A)v 
\end{align*}
Thus we have 
\begin{align*}
	\lambda_1(\tilde A) \leq \lambda_1(A) + \lambda_1(E)
\end{align*}
\begin{theorem}[Weyl]
	If $A,E \in \KK^{I \times I}$ are two Hermitian matrices, then for all k = $1, ..., n$
	\begin{align*}
		\lambda_{k}(A)+\lambda_{n}(E)\leq\lambda_{k}(A+E)\leq\lambda_{k}(A)+\lambda_{1}(E).
	\end{align*}
\end{theorem}

% section Singular Value Decomposition
%%%%%%%%%%%%%%%%%%%%%%%%%%%%%%%%%%%%%%%%%
%%%%%%%%%%%%% SECTION END %%%%%%%%%%%%%
%%%%%%%%%%%%%%%%%%%%%%%%%%%%%%%%%%%%%%%%%

% section basics on probability
%%%%%%%%%%%%%%%%%%%%%%%%%%%%%%%%%%%%%%%%%
%%%%%%%%%%%%% SECTION START %%%%%%%%%%%%%
%%%%%%%%%%%%%%%%%%%%%%%%%%%%%%%%%%%%%%%%%
\section{Basics on Probability}
\subsection{Motivation and single random variables}
% slide 160
\subsubsection*{Motivating example}
Power method for computing singular vectors requires
us to choose a point at random on the sphere. But there are infinitely many points on the sphere
Yet the sphere is too “small” to proceed via a density on the
entire space. We model a random point on $S^1$ using the following random
process. Recall that $S^{n}=\left\{x\in\mathbb{R}^{n+1}:||x||=1\right\}$. \newline
Consider an imaginary player $X$ of darts, with no
experience, throwing darts at random at the two
dimensional disk $\Omega = B_r$ radius $r$ and center $O$.
At every point $\omega \in \Omega$ within the target hit by a dart we
assign a point on the boundary of the target $X(\omega) = \frac{\omega}{||\omega||_2} \in S^1$. $S^1$
can be parameterized by the angle $\theta \in [0,2\pi)$. Hence we consider the player $X$ as
a map from $\omega\in\Omega$ to $\theta$
\begin{align*}
	X: \Omega \rightarrow \RR
\end{align*}
\subsubsection*{Calculating probabilities}
What is the probability of the event 
\begin{align*}
	B = \left\{w\in\Omega:\theta_1\leq X(\omega)\leq \theta_2\right\}?
\end{align*}
%todo finish this part
\subsubsection*{Expectations}
For $N$ attempts of the player the empirical average will be
\begin{align*}
	\frac{1}{N}\sum_{i=1}^N X(\omega_i)
\end{align*}
As $N\rightarrow\infty$ this yields the expectation
\begin{align*}
	\EE X := \frac{1}{2\pi} \int_{0}^{2\pi}\theta d\theta = \pi
\end{align*}
%todo
%\subsubsection*{Densities}
\subsubsection*{Abstract probability theory}
% slide 166
A probability space is given by a triplet $(\Omega,\Sigma,\PP)$. Where the sample space $\Omega$
is the set on which the probability is defined. $\Sigma$ is the $\sigma$-algebra (a family of subsets of $\Omega$)
and defines the admissible events. $\PP$ is the probability measure on $(\Omega,\Sigma)$, that
is, it assigns to each event $B\in\Sigma$ to a value $\in[0,1]$, the probability of the event
through
\begin{align*}
	\PP(B) = \int_B d\PP(\omega) = \int_{\Omega} 1_B(\omega)d\PP (\omega)
\end{align*}
\subsubsection*{The union bound}
Consequence of the properties of a measure: For two disjoint events $B_1,B_2,B_1 \cap B_2 = \emptyset$
\begin{align*}
	\PP(B_1 \cup B_2) = \PP(B_1) + \PP(B_2)
\end{align*}
\begin{theorem}
The union bound (or Bonferroni's inequality, or Boole's
inequality) states that for a collection of events $B_l\in\Sigma$, $l=1,...,n$, we have
\begin{align}\label{eq3_25}
\PP\left(\bigcup_{l=1}^{n}B_{l}\right)\leq\sum_{l=1}^{n}\PP \left(B_{l}\right).
\end{align}
\end{theorem}

\subsubsection*{Proof}
For two sets $B_1$ and $B_2$: Notice that these two sets are equal
\begin{align*}
	B_1 \cup B_2 = \left(B_1 \setminus B_2\right) \cup B_2
\end{align*}
If we write the probabilities of these events 
\begin{align*}
	\PP \left(B_1 \cup B_2 \right) = \PP\left(\left(B_1 \setminus B_2\right) \cup B_2\right)
\end{align*} 
Notice that $\left(B_1 \setminus B_2\right) \cap B_2 = \emptyset$
\begin{align*}
	\PP\left(\left( B_1 \setminus B_2 \right) \cup B_2\right) = \PP \left(B_1 \setminus B_2 \right) + \PP \left( B_2 \right)
\end{align*}
Since $ B_1 \setminus B_2  \subseteq B_1$ we can write the proof in one line as
\begin{align*}
	\PP \left(B_1 \cup B_2 \right)=\PP \left(\left(B_1 \setminus B_2\right) \cup B_2 \right) = \PP \left(B_1 \setminus B_2 \right) + \PP \left( B_2 \right)  \leq \PP(B_1) + \PP(B_2)
\end{align*}
The bound is obtained by the fact $\PP \left(B_1 \setminus B_2 \right) \leq \PP \left(B_1\right)$.
Assume for $n-1$
\begin{align}\label{boole_hyp}
	\PP\left(\bigcup_{l=1}^{n-1}B_{l}\right)\leq\sum_{l=1}^{n-1}\PP \left(B_{l}\right).
\end{align}
Let $\bigcup_{l=1}^{n-1}B_{l} = A$. 
\begin{align}\label{boole_hyp_a}
	\PP\left(A\right)\leq\sum_{l=1}^{n-1}\PP \left(B_{l}\right).
\end{align}
Consider the union $A \cup B_n$
\begin{align*}
	\PP\left(\left(A\setminus B_n\right) \cup B_n \right) = \PP\left(A\setminus B_n\right) + \PP(B_n) 
\end{align*}
Notice that $ \PP \left(A \setminus B_n \right) \leq \PP \left(A\right)$
\begin{align*}
	\PP\left(\left(A\setminus B_n\right) \cup B_n \right) \leq \PP\left(A\right) + \PP(B_n) 
\end{align*}
Using the bound from \cref{boole_hyp_a} we have

\begin{align*}
	\PP\left(\left(A\setminus B_n\right) \cup B_n \right) \leq \PP\left(A\right) + \PP(B_n) \leq \sum_{l=1}^{n-1}\PP \left(B_{l}\right) + \PP(B_n)
\end{align*}
Hence we have
\begin{align*}
	\PP\left(\left(A\setminus B_n\right) \cup B_n \right) = \PP\left(\bigcup_{l=1}^{n}B_{l}\right) \leq \sum_{l=1}^{n}\PP \left(B_{l}\right).
\end{align*}

\subsubsection*{Random variables}
A random variable $X$ is a real-valued measurable function on $(\Omega,\Sigma)$.
\textbf{Recall}: X is called measurable if the preimage
\begin{align*}
	X^{-1}\left(A\right) = \left\{\omega\in\Omega : X(\omega)\in A\right\}
\end{align*}
is contained in $\Sigma$ for all Borel measurable subsets $A\subset \RR $. This means 
reasonable events (values $X$ can take)
are contained in the $\sigma$-algebra.

\subsubsection*{Densities}
The distribution function $F = F_X$ of $X$ is defined as
\begin{align} \label{eq3_26}
	F(t)=\PP (X\leq t),\quad t\in\mathbb{R}.
\end{align}
A random variable X possesses a \textit{probability density}
function $\phi:\RR \rightarrow \RR_+$ if
\begin{align} \label{eq3_27}
	\PP (a < X\leq b)=\int_{a}^{b}\phi(t)\,\mathrm{d}t\quad{\mathrm{~for~all~}}a < b\in\mathbb{R}
\end{align}
Then the density function $\phi = \frac{dF(t)}{t}$. Note that not every random variable has
a density.For example $X$ with $\PP(X=1)=\PP(X=-1)=0.5$. $F_X(t)$ is a partial function and isn't continuous.

%slide 171 Expectations revisited
\subsubsection*{Expectations revisited}
The probability measure associated to a density $\phi$ is then given by
$d\PP  =\varphi(\theta)d\theta$.
Consequently, we can compute for a function g
\begin{align*}
\EE g(X):=\int_{\Omega}{g}(X(\omega))\mathrm{d}\PP (\omega)=\int_{\mathbb{R}}{g}(\theta)\varphi(\theta)d\theta.
\end{align*}
The probability of an event $E=\left\{X \in A\right\}$ satisfies $\PP (E) = \EE 1_E$ and hence 
\begin{align*}
	\PP (E) = \int_A \varphi(\theta)d\theta
\end{align*}
% slide 172
\subsubsection*{Moments}
$\EE X^p$ for $p>0$ are called moments of $X$, while ${\EE|X|}^p$ are called absolute moments.
The quantity $\EE(X-\EE X)^2 = \EE X^2 - (\EE X)^2$ is called variance.
For $1\leq p \leq \infty$, $\left(\EE|X|^p\right)^{1/p}$ defines a norm on the $L^p\left(\Omega,\PP\right)$-space
of all $p$-integrable random variables, in particular, the triangle
\begin{align}\label{eq3_28}
\left(\EE {{|X+Y|}^p}\right)^{1/p} \leq  \left(\EE {{|X|}^p}\right)^{1/p} +  {\left(\EE {|Y|}^p\right)}^{1/p} 
\end{align} 
holds for all $p$-integrable random variables $X$,$Y$ on $\left(\Omega,\Sigma,\PP\right)$. Here, $p$-integrable random variables are random
variables that have bounded $p$-th absolute moments.
\subsubsection*{Important results about random variables}
Hoelder's inequality states that, for random variables $X$, $Y$
on a common probability space and $p,q\geq 1$  with
\begin{align*}
	\frac{1}{p} + \frac{1}{q} = 1
\end{align*}
we have 
\begin{align*}
	| \EE X Y | \leq \left(\EE |X|^{p}\right)^{1/p}\left(\EE |Y|^q \right)^{1/q}  
\end{align*}
Let $X_{n},n\in\mathbb{N}$, be a sequence of
random variables such that $X_n$ converges
to $X$ as $n \rightarrow \infty$ 
in the sense that $\lim_{n\rightarrow\infty}X_n(\omega) = X\left(\omega\right)$
almost surely (a.s.).
\begin{theorem}[Lebesgue's dominated convergence theorem]
	If there exists a random variable $Y$ with $\EE|Y| < \infty$ such that
	$|X_n|<|Y|$ then almost surely $\lim_{n\rightarrow \infty}\EE X_n = \EE X$.
\end{theorem}
\subsubsection*{Moment computations \& Cavalieri's formula}
\begin{proposition}
	\label{prop9.1} 
	The absolute moments of a random variable $X$ can be expressed as
\begin{align} \label{eq3_prop9.1}
\EE |X|^{p}=p\int_{0}^{\infty}\PP (|X|\geq t)t^{p-1} dt \,\quad p > 0.
\end{align}
\end{proposition}
\subsubsection*{Important tool for the proof: Fubini's theorem}
Let $f : A \times B \rightarrow C$ be measurable,
 where $(A, \nu)$ and $(B, \mu)$ are
measurable spaces. If $\int_{A\times B} |f(x, y)| d(\nu \otimes  \mu)(x, y) < \infty $ then 
\begin{align*}
	\int_{ A}\left(\int_{ B}f(x,y)\,d\mu (y)\right)\,d\nu(x)=\int_{ B}\left(\int_{ A}f(x,y)\,d\nu(x)\right)\,d\mu(y)\,.
\end{align*}
\subsubsection*{Proof}
Using Fubini's theorem we derive 
\begin{align*}
	\EE |X|^p = \int_\Omega |X\left(\omega\right)|^p d\PP\left(\omega\right) \\
	= \int_\Omega \int_0^{|X\left(\omega\right)|^p} 1dx \, d\PP\left(\omega\right) \\
	= \int_\Omega \int_0^{\infty} \charf_{|X\left(\omega\right)|^p > x} dx \, d\PP\left(\omega\right) \\
	= \int_0^{\infty} \int_\Omega \charf_{|X\left(\omega\right)|^p > x} d\PP\left(\omega\right) \, dx \\
	= \int_0^{\infty} \PP\left(|X|^p \geq x\right) \, dx 
\end{align*}
Let $t^p = x$, we use the change of variables trick $p \, t^{p-1} \, dt = dx$
\begin{align*}
	\int_0^{\infty} \PP\left(|X|^p \geq x\right) \, dx \\
	= p\int_0^{\infty} \PP\left(|X|^p \geq t^p \right) t^{p-1} \, dt \\
	= p\int_0^{\infty} \PP\left(|X| \geq t \right) t^{p-1} \, dt 
\end{align*}
Hence we have proved \cref{prop9.1}.

\begin{corollary} \label{cor9.2}
	For a random variable $X$ the expectation satisfies
	\begin{align} \label{eq3_cor9.2}
		\EE X=\int_{0}^{\infty}\PP (X\geq t)\,dt-\int_{0}^{\infty}\PP (X\leq-t)\,dt\,.
	\end{align}
\end{corollary}
\subsubsection*{Proof}
Write $X=X_+ + X_-$ where $X_+ = X1_{X\geq 0}$ and $X_- = X1_{X < 0}$. Consequently
$\EE X = \EE X_+ + \EE X_- = \EE |X_+| - \EE |X_-|$. Applying \cref{prop9.1} to both yields (for $p=1$) the \cref{cor9.2}.


\subsubsection*{Tail bounds and Markov inequality}
The function $t \rightarrow \PP\left(|X| \geq t\right)$ is called the \textbf{tail} of $X$. The tail can be estimated by expectations and moments
via the Markov inequality.
\begin{theorem} \label{theorem9.3}
	Let $X$ be a random variable. Then
\end{theorem}
\begin{align} 
	\PP (|X|\geq t)\leq{\frac{\EE |X|}{t}} \, \text{for all} \, t>0. 
\end{align}
\subsubsection*{Proof}
Consider the term
\begin{align} \label{eq_theorem9.2_1}
	t\PP \left(|X| \geq t\right)
\end{align}
since $\PP \left(|X| \geq t\right) = \EE \charf_{|X| \geq t}$ this term \cref{eq_theorem9.2_1}

can be written as
\begin{align} \label{eq_theorem9.2_2}
	t\PP \left(|X| \geq t\right) = t \EE \left[\charf_{|X| \geq t}\right]  = \EE \left[t\charf_{|X| \geq t}\right]
\end{align}
Notice that 
\begin{align*}
	t\charf_{|X| \geq t} \leq |X|
\end{align*}
always holds. This also means
\begin{align*}
	\EE \left[t\charf_{|X| \geq t}\right] \leq \EE |X|
\end{align*}
Notice that left hand side is same as \cref{eq_theorem9.2_1}. Hence we have

\begin{align*}
	t\PP \left(|X| \geq t\right) \leq \EE |X| \\
	\PP \left(|X| \geq t\right) \leq \frac{\EE |X|}{t}.
\end{align*}
\begin{remark}
	As an important consequence we note that for $p > 0$
	\begin{align*}
		\PP (|X|\geq t)= \PP (|X|^{p}\geq t^{p}) \leq t^{-p}\mathbb{E}|X|^{p} \quad \text{for all} \quad t>0
	\end{align*}
	The special case p = 2 is referred to as the \textbf{Chebyshev} inequality.
\end{remark}
\begin{remark}
	For $\theta > 0$ we obtain that for all $t \in \RR$
	\begin{align*}
		\PP (X\geq t)=\PP (\exp(\theta X)\geq\exp(\theta t)) 
		\leq \exp(-\theta t)\mathbb{E}\exp(\theta X)\,.
	\end{align*}
	The function $\theta \rightarrow \EE \exp(\theta X)$ is usually called the 
	\textbf{Laplace transform} or the \textbf{moment generating function} of $X$.
\end{remark}
\subsubsection*{Gaussian Random Variables}
A normally distributed random variable or Gaussian
random variable $X$ has probability density function
\begin{align*}
	\psi(t)=\frac{1}{\sqrt{2\pi\sigma^{2}}}\exp\left(-\frac{(t-\mu)^{2}}{2\sigma^{2}}\right).
\end{align*}
Its distribution is often denoted by $\Nr(\mu, \sigma)$.
It has mean $\EE X = \mu$ and variance $\EE (X - \mu)^2 = \sigma^2$ .
% section basics on probability
%%%%%%%%%%%%%%%%%%%%%%%%%%%%%%%%%%%%%%%%%
%%%%%%%%%%%%% SECTION END %%%%%%%%%%%%%%%
%%%%%%%%%%%%%%%%%%%%%%%%%%%%%%%%%%%%%%%%%

\end{document}
