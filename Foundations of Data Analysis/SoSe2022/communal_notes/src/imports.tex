

\usepackage[left=1.5cm, right=1.5cm, top=2cm, bottom=3cm]{geometry}


\usepackage[main=english]{babel}
\usepackage[T1]{fontenc} % Schrift ordentlich rendern, damit Umlaute auch Umlaute sind
\usepackage[utf8]{luainputenc} % Interpretation des Inputs in utf8 (inkl. Umlaute, Akzente, etc.)
\usepackage{lmodern} % Ordentliche skalierbare Schriftart
\usepackage{blindtext}
\usepackage{dirtytalk}
\usepackage{slashed}
\usepackage{caption} %Abb. statt Abbildung bei Grafiken sowie Tab. statt Table
\usepackage{chngcntr}
\usepackage{xcolor}

\usepackage{framed}
\usepackage{mdframed}

\usepackage{graphicx} %zum einbinden von Grafiken
\usepackage{wrapfig} %zur wrapfigure-Umgebung, damit Grafiken nicht über die ganze Seitenbreite gehen
\usepackage{listings} %to write python code
\usepackage{multicol}
\usepackage{makeidx} 
\usepackage[all]{nowidow}
\usepackage{paralist}
\usepackage{perpage}
\usepackage{tikz}
    \usetikzlibrary{mindmap,decorations.pathreplacing,arrows,calc}
\usepackage[compat=1.1.0]{tikz-feynman}
\usepackage{pgfplots}
    \pgfplotsset{compat = 1.15}
    \def\sothreecirc{\path[thin,dashed,draw=black,fill=black!20] (0,0) circle (\rad)}
    \def\dotmiddle{\fill (0,0) circle (2pt)}
    \def\rad{1.8cm}
\usepackage{todonotes}

\usepackage[separate-uncertainty = true]{siunitx}
\usepackage{nicefrac} %zum darstellen von 1/2 in "schön"
\usepackage{upgreek}
\usepackage{cancel}
\usepackage{braket}
\usepackage{tensor}
\usepackage{marvosym}
\usepackage{mathtools}
\usepackage{mathrsfs} 
\usepackage{amsfonts, amsmath, amssymb} %Mathezeugs
\usepackage{fixmath}
\usepackage{bm}
\usepackage{dsfont}
\usepackage{minted} %to highlight code syntax

\usepackage{stackengine,scalerel}
\usepackage{calc}

\usepackage{amsthm}
\usepackage{hyperref}
\usepackage{cleveref}
\makeatletter
\def\@footnotecolor{myorange}
\define@key{Hyp}{footnotecolor}{%
 \HyColor@HyperrefColor{#1}\@footnotecolor%
}
\def\@footnotemark{%
    \leavevmode
    \ifhmode\edef\@x@sf{\the\spacefactor}\nobreak\fi
    \stepcounter{Hfootnote}%
    \global\let\Hy@saved@currentHref\@currentHref
    \hyper@makecurrent{Hfootnote}%
    \global\let\Hy@footnote@currentHref\@currentHref
    \global\let\@currentHref\Hy@saved@currentHref
    \hyper@linkstart{footnote}{\Hy@footnote@currentHref}%
    \@makefnmark
    \hyper@linkend
    \ifhmode\spacefactor\@x@sf\fi
    \relax
  }%
\makeatother
\hypersetup{
  linkcolor = blue,
  citecolor  = myorange,
  urlcolor   = myblue,
  colorlinks = true,
}
%
\definecolor{myred}  {HTML}{A3061E}
\definecolor{myblue} {RGB} {0,63,119}
\definecolor{myyellow} {cmy} {0,0.263,0.741}
\definecolor{mygreen} {HTML}{0B6E4F}
%
\colorlet{myorange} {myyellow!60!myred}
\colorlet{myviolett} {myred!50!myblue!80}
\renewcommand{\thefootnote}{\roman{footnote}}
\MakePerPage{footnote} %Fußnoten werden am Ende jeder Seite statt am Ende des Dokuments angezeigt


%\setcounter{tocdepth}{1} %Im Inhaltsverzeichnis werden Chapter und Sections angezeigt.

\newtheorem{theorem}{Theorem}[section]
\newtheorem{corollary}[theorem]{Corollary}
\newtheorem{proposition}[theorem]{Proposition}
\newtheorem{lemma}[theorem]{Lemma}
\newtheorem{remark}[theorem]{Remark}
\newlength\shlength

\setlength{\parindent}{0pt}
\setlength{\parskip}{0.5\baselineskip}

\definecolor{shadecolor}{RGB}{192,192,192}

\DeclareMathOperator*{\argmax}{arg\,max}
\DeclareMathOperator*{\argmin}{arg\,min}
